\XAUATsetup{
    ctitlefirst={西安建筑科技大学\LaTeX{}},
    ctitlesecond={学位论文模板v1.0.0},
    etitle={A \LaTeX{} Thesis Template for Xi\textquotesingle an University of Architecture and Technology},
    disciplineclassification={***},
    studentnumber={0987654321},
    cauthor={谢生},
    eauthor={Sheng Xie},
    csupervisor={$\times\times\times$ \hspace{0.5em}教授},
    esupervisor={Prof. $\times\times\times$},
    csubmitdate={20**.04},
    cmajor={***工程},
    emajor={*** Engineering},
    cdefencedate={20**.06},
    edefencedate={June, 20**},
    edepartment={School of ***},
    cfunds={本课题由国家自然科学基金项目(1234567890)和陕西省自然科学基金项目(0987654321)资助。}, % 没有则不填
    efunds={This investigation was supported by National Natural Science Foundation (1234567890) and Shaanxi Provincial Natural Science Foundation (098654321).}, % 没有则不填
}

% 摘要和关键字
\begin{cabstract}  
    论文的摘要是对论文研究内容和成果的高度概括。摘要应对论文所研究的问题及其研究目
    的进行描述,对研究方法和过程进行简单介绍,对研究成果和所得结论进行概括。摘要应
    具有独立性和自明性,其内容应包含与论文全文同等量的主要信息。使读者即使不阅读全
    文,通过摘要就能了解论文的总体内容和主要成果。
  
    论文摘要的书写应力求精确、简明。切忌写成对论文书写内容进行提要的形式,尤其要避
    免“第 1 章……;第 2 章……;……”这种或类似的陈述方式。
  \end{cabstract}
  
  \ckeywords{\TeX, \LaTeX, CJK, 模板, 论文}
  
  \begin{eabstract}
     An abstract of a dissertation is a summary and extraction of research work
     and contributions. Included in an abstract should be description of research
     topic and research objective, brief introduction to methodology and research
     process, and summarization of conclusion and contributions of the
     research. An abstract should be characterized by independence and clarity and
     carry identical information with the dissertation. It should be such that the
     general idea and major contributions of the dissertation are conveyed without
     reading the dissertation.
  
     An abstract should be concise and to the point. It is a misunderstanding to
     make an abstract an outline of the dissertation and words ``the first
     chapter'', ``the second chapter'' and the like should be avoided in the
     abstract.
  \end{eabstract}
  
  \ekeywords{\TeX, \LaTeX, CJK, template, thesis}